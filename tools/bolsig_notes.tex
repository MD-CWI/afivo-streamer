\documentclass[a4paper, 12pt]{article}
\usepackage{graphicx}
\usepackage{subcaption}
\usepackage{float}

\begin{document}

\section*{There are several points to note when using Bolsig+}

	Transport coefficients are affected by the chosen electron density growth condition, which is by default assumed to be exponential in time on online Bolsig+ calculations. When using the offline version of Bolsig+, the user can choose between exponential temporal growth, exponential spatial growth, excitation and attachment not included, and density gradient expansion (see Bolsig+ manual at http://www.bolsig.laplace.univ-tlse.fr/manual.html). The rates in afivo-streamer/transport{\_}data/air{\_}chemistry{\_}v0.txt used exponential temporal growth.

	\begin{figure}[H]
		\centering
		\begin{subfigure}[H]{0.4\textwidth}
			\includegraphics[width=\textwidth]{alpha_diffgrowth.png}
			\caption{Ionization coefficient}
		\end{subfigure}
		\begin{subfigure}[H]{0.4\textwidth}
			\includegraphics[width=\textwidth]{eta_diffgrowth.png}
			\caption{Attachment coefficient}
		\end{subfigure}
		\begin{subfigure}[H]{0.4\textwidth}
			\includegraphics[width=\textwidth]{mu_diffgrowth.png}
			\caption{Mobility coefficient}
		\end{subfigure}
		\begin{subfigure}[H]{0.4\textwidth}
			\includegraphics[width=\textwidth]{d_diffgrowth.png}
			\caption{Diffusion coefficient}
		\end{subfigure}
		\caption{Transport coefficients with different growth conditions}
	\end{figure}
	
	The ionization and attachment coefficients are also calculated by afivo-streamer from the reaction rate coefficients provided in the chemistry file. In the afivo-streamer calculations, the ionization coefficient is affected by the selected growth condition.
    
    \begin{figure}[H]
		\centering
		\begin{subfigure}[H]{0.4\textwidth}
			\includegraphics[width=\textwidth]{alpha_summaries.png}
			\caption{Ionization coefficient}
		\end{subfigure}
		\begin{subfigure}[H]{0.4\textwidth}
			\includegraphics[width=\textwidth]{eta_summaries.png}
			\caption{Attachment coefficient}
		\end{subfigure}
		\caption{Ionization and attachment coefficients computed from reaction rates from data sets with different growth conditions}
	\end{figure}	
	
	If the coefficients calculated by Bolsig+ and the coefficients calculated by afivo-streamer are compared, we see some difference in the ionization coefficient. We also notice that Bolsig+ does not include three-body attachment in its calculation of the attachment coefficient.
	
    \begin{figure}[H]
		\centering
		\begin{subfigure}[H]{0.4\textwidth}
			\includegraphics[width=\textwidth]{alpha_compare.png}
			\caption{Ionization coefficient}
		\end{subfigure}
		\begin{subfigure}[H]{0.4\textwidth}
			\includegraphics[width=\textwidth]{eta_compare.png}
			\caption{Attachment coefficient}
		\end{subfigure}
		\caption{Ionization and attachment coefficients calculated by afivo-streamer and Bolsig+}
	\end{figure}
	
	As of 20/11/2019, including chemistry when using afivo-streamer results to a simulation that employs two ionization coefficients. The photoionization module considers the ionization coefficient calculated by Bolsig+ while afivo-streamer also calculates a separate ionization coefficient from the reaction rate coefficients. These two ionization coefficients do not match.


\end{document}